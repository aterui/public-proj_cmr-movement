\documentclass[11pt, class=article, crop=false]{standalone}
\usepackage[subpreambles=true]{standalone}
\usepackage{newtxtext,newtxmath}
\usepackage{import,
            graphicx,
            parskip,
            url,
            amsmath,
            wrapfig,
            fancyhdr,
            soul,
            tabularx,
            authblk,
            textcomp,
            lineno,
            setspace,
            longtable,
            float}

% side caption figure
\usepackage{sidecap}
\sidecaptionvpos{figure}{t}

% for special characters in bibliography            
\usepackage[utf8]{inputenc}
\usepackage[T1]{fontenc}

% citation setup
\usepackage[euler]{textgreek}
\usepackage[numbers,sort&compress]{natbib}
\setcitestyle{square, comma}
\bibliographystyle{vancouver}

% caption setup
\usepackage[font = small, labelfont = {bf, small}]{caption}
           
% margin
\usepackage[top=2.54cm, bottom=2.54cm, left=2.54cm, right=2.54cm]{geometry}

\linenumbers
\doublespacing

% mini table of contents in appendix
\setcounter{minitocdepth}{2}
\dominitoc

% title
%\title{Body size and local density influence movement patterns in stream fish}
\date{} % remove date from title

% author list
%\author{} % remove author list from main text 

%\maketitle
% title
\title{Body size and local density explain movement patterns in stream fishes}
\date{} % remove date from title

% author list
\author[1]{Ashley LaRoque}
\author[1, 2]{Seoghyun Kim}
\author[1]{Akira Terui}
\affil[1]{Depatment of Biology, University of North Carolina at Greensboro}
\affil[2]{Department of Biological Sciences, Kangwon National University}

\begin{document}

\renewcommand{\theequation}{S\arabic{equation}}
\renewcommand{\thetable}{S\arabic{table}}
\renewcommand{\thefigure}{S\arabic{figure}}

\maketitle
\thispagestyle{empty}

\section*{Contact}
Akira Terui by email at \url{a_terui@uncg.edu}

\newpage
\thispagestyle{empty}

\section*{Keywords}

fish movement, body size, density-dependent, movement drivers, extrinsic and intrinsic, freshwater

\section*{Abstract}

 Movement is a fundamental process in structuring communities, distributing species, and mediating gene flow. Both extrinsic (e.g., density of species) and intrinsic factors (e.g., body size) influence movement patterns, ultimately driving the spatial organization of ecological communities. However, these extrinsic and intrinsic factors are often assessed in isolation, limiting our ability to understand how multiple factors combine to shape movement patterns in nature. Here, we evaluate whether body size (intrinsic) and intra- and interspecific densities (extrinsic) have an impact on the movement rates of four fish species (\textit{Nocomis leptocephalus} bluehead chub, \textit{Semotilus atromaculatus} creek chub, \textit{Lepomis cyanellus} green sunfish, and \textit{L. auritus} redbreast sunfish) in a small stream. We employed a capture-mark-recapture framework to individually track movements, defined as the difference between locations on consecutive (re)captures. We then applied a dispersal-observation model that accounts for detectability, survival, and emigration when inferring movement processes. We found that larger individuals of creek chub and green sunfish were more likely to move, which may be explained by their greater physical ability to balance the energetic cost of moving in tandem with greater competitive ability during settlement. The effect of density on movement was mixed. Green sunfish moved away from a high density of creek chub, but movement declined when bluehead chub density was high. Bluehead chub responded reciprocally to green sunfish, with less movement at high green sunfish density. Movement also declined for creek chubs in the presence of bluehead chub. This may suggest that certain species interact due to predator-prey interactions either directly or indirectly. Collectively, our results suggest that intrinsic (body size) and extrinsic factors (density) influence movement patterns, but their relative importance is species-specific. Further exploring the mechanistic relationship behind drivers of movement will provide greater insight into spatial community dynamics.

\clearpage
\pagenumbering{arabic} 
\newpage

\section{Introduction}

Movement is a ubiquitous characteristic of animals, mediating nutrient exchange, gene flow, and disease spread, among others \citep{cookeMovementEcologyFishes2022, hessDiseaseMetapopulationModels1996, teruiParasiteInfectionInduces2017, fauschLandscapesRiverscapesBridging2002}.
If successful, movers may obtain substantial benefits such as release from intense competition and disease risks \citep{clobertDispersalEcologyEvolution2012, teruiParasiteInfectionInduces2017}.
However, it comes with energetic and opportunity costs \citep{bonteCostsDispersal2012}.
Mobile individuals must make an appropriate decision on when and how they move to improve their fitness, such as growth and survival \citep{bonteCostsDispersal2012}. As a result, movement is influenced by both extrinsic and intrinsic factors with important consequences for the spatial structure of communities \citep{leiboldMetacommunityConceptFramework2004, mcpeekEvolutionPassiveDispersal2024, schlagelMovementmediatedCommunityAssembly2020}. 

Body size exemplifies intrinsic individual conditions that affect movement patterns \citep{clobertDispersalEcologyEvolution2012}. For example, larger individuals tend to move long distances due in part to their greater locomotive capabilities, although the nature of correlation varies greatly among species and ecological contexts \citep{comteEvidenceDispersalSyndromes2018, teruiParasiteInfectionInduces2017, radingerPatternsPredictorsFish2014, debeffeConditiondependentNatalDispersal2012,gilliamMovementCorridorsEnhancement2001, faustinodequeirozEcologicalDriversFish2020}.
Simultaneously, intra- and interspecific population densities -- key extrinsic factors -- influence movement through competitive and mutualistic interactions. For example, competitive interactions may cause individuals to move away from densely populated areas, while mutualistic interactions and higher prey availability may reduce movement rates \citep{thierryInterplayAbioticBiotic2024, rasmussenIndividualMovementStream2017, fronhoferBottomupTopdownControl2018}.

Despite this recognition, the interplay between extrinsic and intrinsic factors on movement has rarely been considered \citep{cookeMovementEcologyFishes2022}. Most previous studies have evaluated these factors in isolation due to field constraints and statistical complexity. However, this makes it difficult to understand how observed movement patterns may emerge in nature because these drivers are not mutually exclusive \citep{mcmahonLinkingHabitatSelection2006}. 
This unified perspective is particularly important, yet challenging to implement in the field, where multiple species interact and may exhibit species-specific responses to both intrinsic and extrinsic factors \citep{teruiNonrandomDispersalSympatric2021}. Addressing this complexity in movement ecology requires a tractable system, where one can directly observe movement processes while maintaining relevance to natural communities.

Stream fishes serve as an excellent model for evaluating how an intrinsic (body size) and extrinsic driver (population density) shape movement patterns. In a small stream, their movement is restricted to a one-dimensional system (either up- or downstream movement), making the direct observation of movement processes highly tractable in the wild. In addition, as individuals of various sizes move across habitat patches, they engage in inter- and intraspecific interactions over space and time \citep{brownHabitatHeterogeneityActivity2010, davidsonSeasonalSpatialHydrological2012, albaneseEcologicalCorrelatesFish2004, nakayamaFinescaleMovementEcology2018}. Thus, fine-scale movements play a pivotal role in shaping behavioral interactions \citep{pikeSimulatingIndividualMovement2023}. This suggests that movement, as it persists in nature, is inherently complex but by evaluating co-occurring drivers, we can better grasp how such patterns prevail. 

Here, we aim to evaluate movement patterns of stream fishes in response to body size and population density. To this end, we conducted a mark-recapture study in a small stream in North Caroline over a four-year period, focusing on four dominant fish species in this system: \textit{Semotilus atromaculatus} creek chub, \textit{Nocomis leptocephalus} bluehead chub, \textit{Lepomis cyanellus} green sunfish, and \textit{L. auritus} redbreast sunfish. 
These species are largely generalist invertivores, with occasional predation upon other fish species and all begin spawning in the spring.
We hypothesize that both intrinsic and extrinsic factors influence the movement of these stream fishes. Specifically, we tested the following predictions:  (1) larger individuals move more often than their smaller counterparts; (2) higher interspecific densities will drive more movement due to competition; and (3) higher intraspecific density will drive more movement among conspecifics.

\section{Method}

\subsection{Study Site and Species}

Our study was conducted in the Piedmont region of North Carolina, USA within a small tributary of the Reedy Fork River located at Gateway Research Park(36.169939 N, 79.722088 W). This stream was considered a second-order perennial from which we selected a 430-m reach for our mark-recapture research. This reach consisted of riffle-pool sequences with its substrate dominated by sand and gravel (Figure \ref{CMR_site}). The downstream boundary of our study reach ended at a concrete bridge, while the upstream boundary ended at a small cascade. 

\begin{figure}[H]
    \centering
    \includegraphics[width=1\linewidth]{output/CMR_site.pdf}
    \caption[Location and schematic of the stream reach.]{Panel A shows the location of the stream reach within the Piedmont of North Carolina, USA. Panel B outlines the reach. Panel C is a schematic of the stream reach. Mark-recapture was conducted in the 430m reach, broken down by 10m sections. Sections consisted of pool-riffle sequences. Each section was sampled using single-pass backpack electrofishing.}
    \label{CMR_site}
\end{figure}

We selected the four most abundant fish species throughout the reach, all of which generally prefer pool habitats and exhibit resident life-history strategies: (64\% in cumulative abundance): two belonging to the Family Leuciscidae (\textit{Semotilus atromaculatus} creek chub (29\% in abundance) and \textit{Nocomis leptocephalus} bluehead chub  (14\% in abundance)) and two belonging to the Family Centrarchidae (\textit{Lepomis cyanellus} green sunfish (14\% in abundance) and \textit{L. auritus} redbreast sunfish (7\% in abundance)). 

The creek chub and bluehead chub are morphologically similar, but differ slightly in their habitat preferences. Creek chub are the most resilient to environmental degradation and have little preference over habitat \citep{bramblettDevelopmentEvaluationFish2005, mccormickDevelopmentIndexBiotic2001}. Bluehead chub are more selective and prefer finer substrate (e.g. pebble) for nest building during the spring spawning season \citep{evelynFacilitationBenthicAssemblages2024}. Despite these subtle differences, habitat requirements of the creek and bluehead chub change across body size such that larger individuals more often occupy areas with greater microhabitat \citep{schlosserPredationRiskHabitat1988, magnanOntogeneticChangesDiel1984}. Both the green and redbreast sunfish are tolerant of a broad range of habitat quality and are nest-builders during spring spawning \citep{wernerCompetitionHabitatShift1977, helmsFeedingGrowthTrophic2018, etnierFishesTennessee1993}. Largely, these four species are generalist insectivores, but they will occasionally prey upon other fish as adults \citep{wernerCompetitionHabitatShift1977, teruiNonrandomDispersalSympatric2021, helmsFeedingGrowthTrophic2018, etnierFishesTennessee1993}. 
% Other water-column and benthic species found (ordered from common to rare) within this reach include: \textit{Etheostoma olmstedi} tessellated darter, \textit{Moxostoma rupiscartes} striped jumprock, \textit{L. macrochirus} bluegill, \textit{Micropterus salmoides} largemouth bass, \textit{Noturus insignis} margined madtom, \textit{Fundulus rathbuni} speckled killifish, \textit{M. collapsum} notchlip redhorse, \textit{Luxilus albeolus} white shiner, \textit{Erimyzon oblongus} creek chubsucker, \textit{Gambusia holbrooki} eastern mosquitofish, \textit{Misgurnus anguillicaudatus} pond loach, \textit{L. gulosus} warmouth, \textit{Clinostomus funduloides} rosy side dace, and \textit{Ameiurus natalis} yellow bullhead. 
Proportional abundance for each of these species can be found in Appendix Figure S1.

\subsection{Fish Sampling}

Mark-recapture sampling took place regularly from November 2020 to August 2024 at an interval of roughly three months (average 93 days), except for one interval (November 2021 to May 2022; 171 days) in which we were unable to conduct a field survey in February 2022 for logistical reasons. As a result, we collected mark-recapture data for 15 occasions.

The study reach was divided into 10-m sections to locate fish at a fine-spatial scale. In each section, fish were collected via single-pass electrofishing (Smith-Root, Inc.) and placed in a five-gallon bucket for handling. All captured individuals were identified to species and measured for total length (mm). Individuals of the study species (> 60 mm in total length) were anesthetized with MS-222 (Tricaine-S) and implanted with a 12-mm passive integrated transponder (PIT) tag to uniquely identify each individual in subsequent recaptures (Oregon RFID). This technique has been described in more detail by Cary et al. \citep{carySurvivalUpperPiedmont2017} and is widely used. 
% A liquid skin glue was used on the incision site for individuals of both chub species to prevent tag loss due to the nature of their scales. 
After the successful implantation, fish were returned to their holding bucket and monitored for about 15 minutes to ensure survival before being released back into their section of capture. Across the 15 occasions and four species, we tagged 2,430 unique individuals Table \ref{tab:capture}.

\subsection{Environmental Variables}

Habitat variables were measured at base-flow conditions along three evenly spaced transects per section. In each transect, we measured the following physical variables at the center and near both sides of the bank, totaling nine measurements per section: water depth (nearest cm), current velocity (m/s), and dominant substrate type (silt, <0.1 mm; sand, 0.1--2 mm; gravel, 2--16 mm; pebble, 16--64 mm; cobble, 64--256 mm; boulder, 256--512 mm; bedrock, >512 mm). In addition, we measured the total aerial coverage of habitat refuge areas (HRA; m$^2$) like undercut banks and woody debris per section as these structures may represent important microhabitats for the study species. We calculated the surface area of each section as the mean wet width (measured at each transect) times the section length. Average habitat metrics can be found in Table \ref{tab:habitat}. 
% Each habitat metric was scaled by occasion. Scaled velocity, depth, substrate, and HRA were not strongly correlated with one another (Pearson's $r < 0.40$). However, we chose not to include depth because it was weakly correlated with HRA (Pearson $r = 0.39$), and we did not include substrate because it was a visual estimate subject to bias. Therefore, we included scaled velocity and HRA as environmental covariates to control for their effects on movement. 

% latex table generated in R 4.4.1 by xtable 1.8-4 package
% Wed Oct 22 14:46:50 2025
\begin{table}[ht]
\centering
\caption{Mean and standard deviation across sections and occasions for each habitat variable.} 
\label{tab:habitat}
\begin{tabular}{lrr}
  \hline
Habitat Metric & Mean & Standard Deviation \\ 
  \hline
Mean Section Area (m$^2$) & 32.51 & 9.21 \\ 
  Pool Area (m$^2$) & 18.60 & 17.14 \\ 
  Riffle Area (m$^2$) & 4.84 & 8.35 \\ 
  Run Area (m$^2$) & 9.07 & 11.21 \\ 
  Habitat Refuge Area (m$^2$) & 0.45 & 0.84 \\ 
  Mean Depth (cm) & 18.78 & 10.31 \\ 
  Section Length (m) & 9.94 & 1.07 \\ 
  Mean Substrate (mm) & 4.20 & 1.42 \\ 
  Mean Velocity (m/s) & 0.06 & 0.08 \\ 
  Mean Width (m) & 3.26 & 0.82 \\ 
  Mean Temperature ($^\circ$C) & 14.58 & 6.27 \\ 
   \hline
\end{tabular}
\end{table}


\subsection{Statistical Analysis}

We used the dispersal-observation model to evaluate the effects of both intrinsic and extrinsic variables on movement behaviors \citep{teruiModelingDispersalUsing2020, teruiNonrandomDispersalSympatric2021, teruiParasiteInfectionInduces2017, rodriguezRestrictedMovementStream2002, fujiwaraEstimationDispersalKernels2006}. This modeling approach fundamentally differs from ordinary linear models (e.g., generalized linear models) in two key ways. First, it integrates movement and observation processes into a single model. In field studies, observed movement is often confounded by imperfect detection and unmeasured emigration or mortality. The dispersal-observation model yields less biased estimates of movement parameters by explicitly accounting for these factors \citep{teruiModelingDispersalUsing2020}. Second, this framework models the \textit{dispersion} parameter of the movement kernel, i.e., the spread of movement distances, as a function of linear predictors. This contrasts with conventional linear models, which typically relate predictors to the \textit{mean} of a probability distribution. As a result, this approach more appropriately captures variation in movement behavior across ecological contexts \citep{teruiModelingDispersalUsing2020, pepinoFishDispersalFragmented2012, rodriguezRestrictedMovementStream2002}.

\textit{Movement process}. We define movement as fine-scale shifts over habitat patches that occur between consecutive recapture events (i.e., movement between occasion $t$ and $t+1$). Let $X_{1,i}$ and $X_{0,i}$ denote locations of recapture at occasion $t+1$ and capture at occasion $t$ for individual replicate $i$, which were measured as the distance from the midpoint of the section to the downstream end of the study stretch. We assumed $X_{1,i}$ as a random draw from a Student-t distribution conditional on the capture location $X_{0,i}$ as:    

\begin{equation}
    X_{1, i}|X_{0, i}, \sigma_i \sim \text{t}(X_{0, i}, \sigma_i^2, \nu)
    \label{eq:student-t}
\end{equation}

where $\sigma_i$ is the standard deviation describing the distance moved between consecutive capture and recapture occasions ($X_{1,i} - X_{0,i}$), and $\nu$ is the degree of freedom.
We chose a Student-t distribution because movement data are often heavy-tailed \citep{clobertDispersalEcologyEvolution2012}.
The degrees of freedom parameter, $\nu$, controls the extent of heavy-tailedness, with lower values indicating fatter tails. As $\nu$ approaches infinity, the Student-t distribution converges to a normal distribution. Here, we selected a moderate value of $\nu = 5$ to ameliorate the effect of outliers, such that the coefficient estimates of linear predictors remain relatively insensitive to them (Equation \ref{eq:linear-pred}) \citep{lunnBUGSBookPractical2012}.

We linked the standard deviation of  movement distance to predictors using a log-link function: 

\begin{equation}
    \ln \sigma_i = \beta_0 + \sum_{k} \beta_k x_{k,i} + \ln \eta_i
    \label{eq:linear-pred}
\end{equation}

where $\beta_0$ is the intercept and $\beta_k$ is the regression coefficient for the $k$-th predictor $x_{k,i}$. The log-transformed time interval between capture and recapture $\ln \eta_i$ [ln day] was included as an offset term to standardize the movement duration between observations.

Our model included the following predictors at capture ($x_k$ in Equation \ref{eq:linear-pred}): body size, weighted densities of dominant fish species (bluehead chub, creek chub, green sunfish, and redbreast sunfish), Julian day, HRA, and current velocity.
We included body size at capture as a proxy variable for the individual’s locomotive capacity (internal factor), whereas the weighted fish densities were included to evaluate intra- and interspecific density-dependence in movement (external factors).
Fish densities at the capture occasion were weighted by distance to the capture location to reflect that fish individuals in close proximity may have a greater influence on movement.
We performed this distance weighting because fish individuals might sense fish densities within a certain distance range, perhaps beyond the capture section.
Specifically, the distance-weighted fish density at section $s$, denoted $D^{(w)}_s$, was calculated as $D^{(w)}_s = \sum_{s'} D_{s'}\exp(-\lambda d_{ss'})$, where $D_{s'}$ is the fish density corrected for season-specific imperfect detection (see Appendix, Detection Model and Table S1), $d_{ss'}$ is the separation distance between the midpoints of sections $s$ and $s'$, and $\lambda$ is a rate parameter that defines how quickly the density weight decays with increasing separation distance. 
We set $\lambda = 0.1$, indicating that density influences are strongest within approximately 10 meters, while still allowing for non-negligible contributions from greater distances.
We chose this value because previous studies have consistently shown that stream fishes typically remain within 10–20 meters over extended periods, although some individuals do disperse over longer distances \citep{skalskiModelingDiffusiveSpread2000, pepinoFishDispersalFragmented2012, rodriguezRestrictedMovementStream2002, teruiNonrandomDispersalSympatric2021, radingerPatternsPredictorsFish2014}.

Julian day was included to account for potential seasonality in movement patterns. For example, it may capture, albeit implicitly, increased movement due to seasonal spawning behaviors. HRA and current velocity were incorporated to control for possible effects of microhabitat structure variation between sections. HRAs may characterize microhabitats that harbor a considerable number of individuals. Current velocity distinguishes pools, runs, and riffles. The inclusion of these predictors allows the model to evaluate the effects of body size and fish densities after accounting for potential influences of confounding factors.
We normalized HRA and current velocity by dividing them by their cross-sectional means calculated for each occasion.
This normalization helps properly account for spatial habitat effects on movement processes by removing seasonal differences in their means.

All predictors except for current velocity and HRA were standardized to a mean $0$ and a standard deviation $1$ prior to the statistical analysis.

\textit{Observation process}. Our capture-recapture data is the imperfect representation of movement processes because fish are recaptured only when the following conditions are satisfied simultaneously: alive, stay, and detected in the study section. Our observation model accounts for this process by describing the recapture state $Y_i$ (1 if recaptured 0 otherwise) as random draws from a Bernoulli distribution:

\begin{equation}
    Y_i \sim \text{Bernoulli}(\phi z_i)
\end{equation}

where $\phi$ is the product of survival, detection, and tag-retention probabilities (hereafter, ``recapture'' probability) and $z_i$ is the binary latent variable indicating whether individual $i$ stayed in the study section at the recapture occasion.
The latent variable $z_i$ was determined by the observed (if recaptured) or predicted location (if not recaptured) of individual replicate $i$ as: 

\begin{equation}
    z_i =
    \begin{cases}
        1~\text{if}~0 \le X_{1,i} \le L~\text{(stay)},\\
        0~\text{otherwise (emigrate)}.
    \end{cases}
\end{equation}

$L$ is the upstream terminal of the study reach ($L = 430$). When $X_{1,i}$ was unobserved (i.e., not recaptured), a predicted value was drawn from Equation \ref{eq:student-t} through the Markov Chain Monte Carlo simulations (see below), thus accounting for the observation process when estimating the movement parameter $\sigma_i$.
This coupling of observation and movement processes accounts for permanent emigration and yields less biased estimates of movement parameters \citep{teruiModelingDispersalUsing2020}.
In addition, by including unrecaptured individuals as a recapture state ($Y_i = 0$), the model can still use the partial movement information from those not recaptured.

The model was fitted to the data for each species separately using JAGS version 4.3.1 \citep{plummerJAGSProgramAnalysis2003}. Weakly informative priors were assigned to parameters: $\text{Normal}(0, 2.5^2)$ for intercept $\beta_0$, $\text{Normal}(0, 1^2)$ for coefficients $\beta_k$, $\text{Trunc}(\text{Normal}(0.5, 1^2), 0, 1)$ for recapture probability $\phi$, and five for the degree of freedom $\nu$. Markov chain Monte Carlo (MCMC) simulations were run for 40,000 iterations with a 15,000 burn-in period and we retained 1,000 samples per chain by thinning every 30 steps to calculate posterior probabilities. Model convergence was checked by ensuring that the potential scale reduction factor, referred to as R-hat, was less than 1.1 for all parameters. All statistical analyses were conducted in R version 4.4.0 \citep{rcoreteamLanguageEnvironmentStatistical2021}.

\section{Results}

Across 15 sampling occasions, we tagged 2,430 unique individuals of our target species, 273 of which were recaptured at least once in consecutive surveys (Table \ref{tab:capture}). Absolute movement distances ranged from 0 to 390 m across species and seasons (Appendix Figure S2). For tagged individuals, redbreast sunfish exhibited the largest mean body size ($96.2 \pm 24.4$ mm), followed by creek chub ($92.4 \pm 21.1$ mm), bluehead chub ($91.4 \pm 21.0$ mm), and green sunfish ($84.6 \pm 18.4$ mm) (body size of recaptured compared to unrecaptured can be found in the Appendix Figure S3).
Densities were highest among the creek chub ($0.62 \pm 0.56$ individuals/m$^2$), followed by green sunfish ($0.51 \pm 0.50$), bluehead chub ($0.31 \pm 0.34$), and redbreast sunfish ($0.29 \pm 0.49$).  

% latex table generated in R 4.4.1 by xtable 1.8-4 package
% Tue Jan 14 20:06:29 2025
\begin{table}[ht]
\centering
\caption{The number of unique individuals, unique recaptures, captured replicates, and recaptured replicates listed from left to right for each target species.} 
\label{tab:capture}
\begin{tabular}{lrrrr}
  \hline
Species & Unique & Unique Recaptures & Replicate Captures & Replicate Recaptures \\ 
  \hline
Bluehead chub & 479 &  73 & 615 &  82 \\ 
  Creek chub & 1191 & 111 & 1391 & 145 \\ 
  Green sunfish & 503 &  56 & 612 &  60 \\ 
  Redbreast sunfish & 257 &  33 & 317 &  37 \\ 
   \hline
\end{tabular}
\end{table}


\begin{figure}
    \centering
    \includegraphics[width=0.9\linewidth]{output/fig_est.pdf}
    \caption{Posterior distributions for parameters in the dispersal-observation model. Median estimates are denoted by points with error bars showing the traditional 95\% credible interval. 
    Distributions are colored in proportion to the posterior probability of each parameter (approximated as the proportion of MCMC samples above or below zeros).
    Higher values of the posterior probability indicate statistically robust effects.
    Note that our model also included Julian day, habitat refuge area, and current velocity to minimize potential bias in estimating the effects of body size and fish densities.
    Full parameter estimates are provided in Table S2.}
    \label{fig:fig_est}
\end{figure}

Our model analysis revealed that movement responses to body size and density are highly species-specific (Figure \ref{fig:fig_est}). Creek chub and green sunfish increased movement distance with their body size, and their posterior probabilities were greater than 0.95 (Figures \ref{fig:fig_est} and \ref{fig:fig_size}; see also Appendix Table S2). 
However, the movement patterns of bluehead chub and redbreast sunfish showed vague relationships with body size, with the posterior distributions centered around zero (Figure \ref{fig:fig_est}).

\begin{figure}
    \centering
    \includegraphics[width=0.75\linewidth]{output/fig_size.pdf}
    \caption{Influence of body size (total body length at capture) on movement distance. Panels and colors distinguish species. Points represent observed values for each recaptured individual. Dark and light shades indicate movement ranges that are predicted to encompass 50\% and 90\% of individuals by the observation-dispersal model. Shades are not shown when the posterior probability of the body size effect was less than 0.95. Full parameter estimates are provided in Table S2.}
    \label{fig:fig_size}
\end{figure}

Interspecific density-dependence in movement was highly species-specific (Figure \ref{fig:fig_est} and Figure \ref{fig:fig_density}). Green sunfish responded to both the creek and bluehead chub, but in contrasting ways. Posterior distributions indicated that the green sunfish responded positively to the density of creek chub, suggesting that these species moved away from areas with high population density of creek chub. In contrast, green sunfish responded negatively to bluehead chub such that movement declined in areas with a high population density of bluehead chub. Similarly, posterior distributions indicated that bluehead chub responded negatively to green sunfish density causing movement to decline when green sunfish density was high. Lastly, creek chub movement was reduced when bluehead chub density was high. Interestingly, no species showed strong evidence of intraspecific density dependence in movement (Figure \ref{fig:fig_est} and \ref{fig:fig_density}).

\begin{figure}
    \centering
    \includegraphics[width=0.8\linewidth]{output/fig_density.pdf}
    \caption{Influence of density on movement distance. Rows represent species responding to intra- and interspecific densities, while columns indicate the species whose population densities influence movement. Each panel describes the absolute movement of a species in response to the density of a given species. Colors represent response species. Diagonal panels show the response to intraspecific density, while the off-diagonal panels indicate the response to interspecific densities. Points represent observed values for each recaptured individual. Dark and light shades indicate movement ranges that are predicted to encompass 50\% and 90\% of individuals by the observation-dispersal model. Shades are not shown when the posterior probability of the density effect was less than 0.95. Full statistics are provided in Table S2.}
    \label{fig:fig_density}
\end{figure}

The effect of Julian day on movement was only present in the green sunfish, suggesting movement declines as the year progresses into fall and winter. The effects of environmental variables (HRA and velocity) were significant only in redbreast sunfish, with their movement declining as HRA and current velocity increased. 
Creek chub also tended to move less when current velocity was high, but this effect was marginal.
These results were summarized in the Appendix Table S2.

\section{Discussion}

Movement is shaped by species responses to extrinsic and intrinsic drivers \citep{clobertDispersalEcologyEvolution2012}. Yet, these factors are often studied in isolation, causing a lack of quantitative analysis comparing their relative influences. Our four-year data set of mark-recapture research unveiled that both intrinsic (body size) and extrinsic (population densities) factors influence stream fish movements, but their influences varied among species. While this makes our findings difficult to generalize, species-level movement responses can provide greater insight into how species interactions are distributed through non-random movement. 

Our first prediction of size-dependent movement was supported by creek chub and green sunfish. Movement requires great energetic expenditure that needs to be balanced among an individual’s needs \citep{boisclairImportanceActivityBioenergetics1989, joblingBioenergeticsFeedIntake1993, cookeMovementEcologyFishes2022}. Body size is one such limiting factor in determining metabolism \citep{beamish2SwimmingCapacity1978, rubio-graciaSizerelatedEffectsInfluence2020}. Because larger individuals have lower relative maintenance costs than smaller individuals due to greater lipid reserves \citep{brownSizeMattersTest2004, krauseRefugeUseFish1998, kannoBodyConditionMetrics2023}, they may be better able to balance the cost of moving \citep{schlagelMovementmediatedCommunityAssembly2020}, such as traversing across stream sections with swift currents. Not only are larger individuals at an energetic advantage, but they may serve as a stronger competitor in the settlement phase \citep{rasmussenIndividualMovementStream2017}. 
However, size-dependent movement patterns can be highly flexible and system-specific. For example, in South Carolina streams during high-flow disturbance, creek chub did not exhibit size-dependent movement, whereas bluehead chub -- despite showing no such pattern in the present study -- demonstrated increased movement distance with body size \citep{teruiNonrandomDispersalSympatric2021}. Therefore, species that showed weak or no relationship between movement and body size in this study (bluehead chub and redbreast sunfish) might still exhibit size-dependent movement in other stream systems.
Comparative studies across multiple stream systems may offer deeper insights into the conditions under which size-dependence emerges.

Our second prediction -- higher interspecific density drives more movement -- was supported only by the green sunfish, which moved longer distances as the density of creek chub increased. The increased movement of green sunfish by creek chub may be indicative of competition between species. In small streams, like the one utilized here, the diets of our target species are known to overlap significantly \citep{collarPISCIVORYLIMITSDIVERSIFICATION2009, lemlySuppressionNativeFish1985, karrAssessmentBioticIntegrity1981, leonardApplicationTestingIndex1986}, potentially explaining the mechanism behind our observation. As a consequence of this movement process, the green sunfish may move to potentially suboptimal habitats in order to avoid this type of interaction \citep{jacobHabitatChoiceMeets2018, thierryInterplayAbioticBiotic2024}. The green sunfish in particular may be successful in this approach as they are a hardy species that balances the cost and benefit of movement well \citep{lemlySuppressionNativeFish1985, moyleInlandFishesCalifornia2002}. However, creek chub did not move away from areas with a higher density of green sunfish, suggesting that this interspecific interaction through movement is asymmetric. This result implies, albeit preliminarily, that creek chub are competitively superior to green sunfish in our study stream.

Contrary to our expectations, however, multiple species tended to stay in areas when interspecific density was high. This pattern was evident for the green sunfish and creek chub when density of bluehead chub was elevated, as well as for the bluehead chub when green sunfish density was elevated.
The exact mechanism is unclear, but predator-prey interactions might underlie this interspecific density-dependence \citep{jacobHabitatMatchingSpatial2015}. 
Green sunfish have been documented to prey upon various chub species, including bluehead chub \citep{lemlySuppressionNativeFish1985, borrelliPuttingLakeTogether2023}. In this case, small individuals of bluehead chub may be susceptible to predation by the green sunfish. Additionally, bluehead chub build large nests during spawning which create community 'hotspots' drawing a diverse array of organisms nearby \citep{tumoloResourceModificationEcosystem2023, boltonRecognizingGapeLimitation2015}. These bluehead chub nests are built of gravel and pebbles which can alter the microhabitat of an area providing greater area not only for fish species, but for aquatic macroinvertebrates as well. Thus, a higher density of bluehead chub may indicate greater food availability, potentially discouraging movement of the green sunfish as well as the creek chub. Yet, experimental approaches are warranted to elucidate biological causality. 

The decreased movement of bluehead chub in response to green sunfish is puzzling, and there are multiple possible explanations for this pattern.
First, smaller green sunfish could serve as prey for larger bluehead chub. Yet, it is difficult to envision them preferentially targeting sunfish, which possess spiny fins, over other potential prey items. 
Second, both species might respond to shared environmental cues, such as increased microhabitat availability \citep{careyEffectsLittoralHabitat2010}.
For example, creek chub and redbreast sunfish both traveled shorter distances with increasing current velocity and/or HRA, suggesting that microhabitat suitability can influence movement decisions. However, while this possibility should not be disregarded, neither of these environmental variables was significant in bluehead chub and green sunfish. Lastly, green sunfish might have facilitative interactions with bluehead chub. Green sunfish build nests by fanning their fins over substrates \citep{thorpPatchesResponsesLake1988}. This spawning behavior may increase access to nest-building materials for bluhead chub. At present, our data is not sufficient to identify likely mechanisms, and controlled experiments should help elucidate this enigmatic relationship.

Interestingly, no species responded significantly to intraspecific densities, countering our third prediction. This result is somewhat surprising given the wealth of studies showing intraspecific competition exceeding interspecific competition in fishes \citep{websterMechanismsIndividualConsequences2000, wardIntraspecificFoodCompetition2006} and other taxa \citep{adlerCompetitionCoexistencePlant2018, barabasEffectIntraInterspecific2016, thompsonProcessbasedMetacommunityFramework2020, chessonRolesHarshFluctuating1997, tilmanResourceCompetitionCommunity1982, mcpeekIntraspecificDensityDependence2012}. While high conspecific density could impose intensive intraspecific competition, it also increases the likelihood of finding potential mates, dilution effects, and group feeding (e.g., Allee effects) \citep{courchampAlleeEffectsEcology2008, gascoigneAlleeEffectsDriven2004, teruiCrypticAlleeEffect2015}. In our system, this may overshadow the relative competition for other resources like habitat and food, thus producing stronger interspecific interactions. Although further studies are needed to illuminate underlying mechanisms, these opposing influences of conspecific density could potentially obscure the influence on movement patterns. 

Although our study illuminated interesting patterns, the results must be viewed with caution. First, while we selected extrinsic and intrinsic variables representative of individual movement ability and species interactions, other potential variables (e.g., sex, disturbance) may be playing a role in driving movement.
Evaluating movement as it occurs \textit{in situ} is an ambitious task; as a consequence, not all possible drivers of movement could be evaluated, like any other field study.
This limitation could explain why one of the study species, bluhead chub, did not show clear movement patterns in response to selected intrinsic and extrinsic variables.
Combining field research with controlled experiments may help address this limitation \citep{fronhoferBottomupTopdownControl2018, nathanMovementEcologyParadigm2008}.
Second, we experienced relatively low consecutive recapture rates, likely because of the small spatial extent of our study area. Our dispersal-observation model is best suited for this type of data though, as it accounts for emigration from the 430-m study reach, imperfect detection, and survival processes to minimize statistical biases arising from this limitation \citep{teruiModelingDispersalUsing2020}.
Our simulation study found that this modeling approach provides unbiased estimates of movement parameters, even if the proportion of recaptures is low, by explicitly modeling imperfect recapture.
Yet, low recapture rates increase the uncertainty of parameter estimates \citep{teruiModelingDispersalUsing2020}.
Although our results should be qualitatively robust, their effect size must be interpreted carefully.
Lastly, we observed ``outlier'' movers in the data (Figures \ref{fig:fig_size} and \ref{fig:fig_density}). 
However, since our model employed a Student-t distribution (see Equation \ref{eq:student-t}), our statistical inference is insensitive to these outlier values (i.e., conservative) \citep{lunnBUGSBookPractical2012}. 
Nevertheless, it is important to recognize that long-distance movers can play critical roles in ecological processes \citep{clobertDispersalEcologyEvolution2012}.
Future studies addressing this limitation would be particularly valuable in advancing movement research.

Movement mediates how species interact, shaping the dynamics of spatially-structured communities \citep{schlagelMovementmediatedCommunityAssembly2020}. Our study demonstrates how intrinsic (e.g., body size) and extrinsic (e.g., population density) drivers can help pinpoint potential movement patterns in freshwater fishes, with variation among species elucidating a potential avenue to explore community dynamics in the future. Extrapolating how non-random movements, such as those identified in this study, affect species coexistence and the spatial patterns of community organization will be essential in connecting behavioral ecology to spatial theory.

\bibliography{tex/references}

\newpage

\section*{Data Accessibility}
All data and R code can be found on the GitHub repository under the following link: \url{https://github.com/ashleylaroque/public_cmr_movement}

\section*{Competing Interests}

None declared.

\section*{Author Contributions}

A.T. and A.L. conceived the idea; A.L., S.K., and A.T. performed  field work; A.L. and A.T. performed statistical analysis and drafted the manuscript; All authors contributed to revision and gave final approval for  publication.

\section*{Acknowledgments}

This work is based on materials supported by University of North Carolina at Greensboro Startup, the Faculty Internal Award, John O'Brien Ecological Field Award, and Dorothy Levis Munroe Research Award funds. We graciously thank our undergraduate students for their help in conducting seasonal fieldwork and digitizing data. We are grateful to University of North Carolina at Greensboro and North Carolina Agricultural and Technical State University for their corroboration in granting us access to Gateway Research Park. 

\newpage
\appendix



\section{Detection Model}

Our capture-recapture data was collected seasonally and fish may be subject to different detection probabilities by season.
As such, accounting for detection probabilities is essential to obtain reliable estimates of fish density.
We utilized a spatial Cormack-Jolly-Seber (CJS) model \citep{schaubEstimatingTrueInstead2014} to account for seasonal detection probabilities. 
This model allowed us to estimate seasonal detection probabilities while accounting for permanent emigration and survival.

\textit{Observation process} -- 
We assumed that the recapture state $Y_{j,t}$ for unique individual $j$ at occasion $t$ ($Y_{i,t} = 1$ if recaptured, $0$ otherwise) is a random draw from a Bernoulli distribution:

\begin{equation}
    Y_{j,t} \sim \text{Bernoulli}(p_{i,t} r_{j, t} z_{j,t}),
\end{equation}

where $p_{j,t}$ is the detection probability, $r_{j, t}$ is the binary latent variable indicating the survival state of individual $j$, and $z_{j, t}$ is the binary latent variable indicating whether individual $j$ remained in the study section.

Our primary interest was to estimate seasonal detection probabilities.
We allowed $p_{j, t}$ to vary by season in a logit scale:

\begin{equation}
    \text{logit}~p_{j,t} = \mu_p + \alpha_p Q_{t},
\end{equation}

where $Q_{t}$ is a dummy variable discerning winter (coded as 0: November -- February) and summer (coded as 1: May -- August).
This formulation translates into the winter detection as $p_{\text{win}} = \text{inv.logit}(\mu_p)$ and the summer detection as $p_{\text{sum}} = \text{inv.logit}(\mu_p + \alpha_p)$

% Our ability to detect individuals that have remained within the study section is partially dependent on seasonal changes such as day length and temperature.
% Therefore, we accounted for this difference in our detection probability $p_{i,t}$ across individuals $i$ and occasions $t$ using a logit-link function:

% \begin{equation}
%     \text{logit}(Ym_{i, t}) \sim \mu \alpha Sm_{i, t}
% \end{equation}

% where $\mu$ is the prior for recapture, $\alpha$ is the prior that accounts for the seasonal effect, and $Sm$ is the seasonal index broken into two components (0 = winter, 1 = summer). 

\textit{State process} -- 
Our model accounted for survival and movement processes to obtain less biased estimates of detection probabilities.
The survival state in the (spatial) CJS model conditions on the first capture of individual $j$.
Let $f_{j}$ denote the first capture occasion for individual $j$.
Then,

\begin{align}
    \begin{split}
        r_{j, f_j} &= 1,\\
    r_{j, t + 1} | r_{j, t} &\sim \text{Bernoulli}(r_{j,t} s_{j,t}),    
    \end{split}
\end{align}

where $s_{j,t}$ is the survival probability between $t$ and $t + 1$.
Since our interest was to estimate detection probabilities, we made a simplifying assumption on the survival probability as $s_{j, t} = \psi^{\eta_{j,t}}$, where $\psi$ is the daily survival probability and $\eta_{j,t}$ is the interval (unit: day) between occasion $t$ and $t + 1$.

We also accounted for emigration.
Let $X_{j, t+1}$ and $X_{j, t}$ denote locations of recapture at occasion $t+1$ and capture at occasion $t$ for unique individual $j$, which were measured as the distance from the midpoint of the section to the downstream end of the study stretch.
As in the main text, we assumed $X_{j,t+1}$ as a random draw from a normal distribution conditional on the capture location $X_{j,t}$ as:

\begin{align}
    \begin{split}
        X_{j, t+1}|X_{j, t}, \sigma_{j, t} &\sim \text{Normal}(X_{j, t}, \sigma_{j, t}^2)\\
        \ln \sigma_{j, t} &= \ln \sigma_0 + \ln \eta_{j, t}
    \end{split}
    \label{eq:si-normal}
\end{align}

where $\sigma_0$ is the standard deviation describing the daily distance moved between occasion $t$ and $t+1$ and $\eta_{j,t}$ is the interval (unit: day) between the occasions for individual $j$.
Unlike the main movement model, we were unable to include predictors for $\sigma_{j,t}$ since predictor values were unavailable when individual $j$ was not recaptured.

The latent variable $z_{j,t}$ was determined by the observed (if recaptured) or predicted location (if not recaptured) of unique individual $j$ as: 

\begin{equation}
    z_{j,t} =
    \begin{cases}
        1~\text{if}~0 \le X_{j,t} \le L~\text{(stay)},\\
        0~\text{otherwise (emigrate)}.
    \end{cases}
\end{equation}

$L$ is the upstream terminal of the study reach ($L = 430$).
When $X_{j, t+1}$ was unobserved (i.e., not recaptured), a predicted value was drawn from Equation \ref{eq:si-normal} through the Markov Chain Monte Carlo simulations.

% \textit{Survival probability}- Individuals must survive from the previous occasion in order to possibly be recaptured. To estimate the probability $\phi$ of an individual $i$ surviving to occasion $t$, the following is assumed:

% \begin{equation}
%     \text{logit}(\phi.day_{i, t}) \sim \mu_\phi
% \end{equation}

% where the survival probability per day $\phi.day$ is described using a logit-link function. Here, $\mu_\phi$ is the prior for survival. This is used to described the cumulative probability of survival $\phi$ using a log-link function: 

% \begin{equation}
%     \text{log}(\phi_{i, t}) \sim Int_{i, t} \text{log}(\phi.day_{i, t})
% \end{equation}

% where $Int_i,t$ is the time interval between capture and recapture in accordance with log-transformed survival per day $\phi.day_{i, t}$.

The model was fitted to the data for each species separately using JAGS \citep{plummerJAGSProgramAnalysis2003}. Vague or weakly informative priors were used for parameters: $\text{Unif}(0, 1)$ for survival probability $s_{j,t}$, and $\text{Normal}(0, 10^2)$ for the seasonal effect $\alpha_p$ and the daily movement parameter $\ln \sigma_0$. Markov chain Monte Carlo (MCMC) simulations were run for 20,000 iterations with a 1,000 burn-in period and we retained 1,000 samples per chain by thinning every 40 steps to calculate posterior probabilities. Model convergence was checked by ensuring that the potential scale reduction factor, referred to as R-hat, was less than 1.1 for all parameters. All statistical analyses were conducted in R version 4.4.0 \citep{rcoreteamLanguageEnvironmentStatistical2021}. 

Estimated detection probabilities were summarized in Table \ref{tab:detection}. We calculated the corrected density $N_{\text{cor}}$ as $N_{\text{cor}} = N_{\text{obs}}p_{\text{sum}}^{-1}$ for summer months and $N_{\text{cor}} = N_{\text{obs}}p_{\text{win}}^{-1}$ for winter months, where $N_{\text{obs}}$ is the observed fish density calculated by section and species as the number of individuals per unit surface area. $N_{\text{cor}}$ was used as predictors in our movement model in the main text.

\newpage

\section{Figures}

\subsection{Proportional Abundance}

\begin{figure}
    \centering
    \includegraphics[width=1\linewidth]{output/fig_abundance.pdf}
    \caption{Percent abundance of each species collected during backpack electrofishing ordered from most to least abundant.}
    \label{fig:fig_abundance}
\end{figure}

\newpage

\subsection{Absolute Movement}

\begin{figure}
    \centering
    \includegraphics[width=1\linewidth]{output/fig_total_move.pdf}
    \caption{The frequency of absolute movement for each target species shown for both winter and summer seasons. Fine-scale movement is prevalent despite seasonality.}
    \label{fig:fig_total_move}
\end{figure}

\newpage

\subsection{Median Length}

\begin{figure}
    \centering
    \includegraphics[width=1\linewidth]{output/fig_size_dist.pdf}
    \caption{Median length at initial capture for each target species described by the solid black line within the boxplot. Colors indicate whether it is either recaptured (yes) or non-recaptured (no) individuals. The interquartile range is depicted by the size of each box. Outliers are shown as points.}
   \label{fig:fig_size_dist}
\end{figure}

\newpage

\section{Tables}

\subsection{Seasonal detection probabilities}

\begin{table}[ht]
\centering
\caption{Seasonal detection probabilities estimated using the spatial Cormack-Jolly-Seber (CJS) model, presented as median estimates with corresponding 95% credible intervals.} 
\label{tab:detection}
\begin{tabular}{lll}
  \hline
Species & Season & Estimate \\ 
  \hline
Bluehead chub & Winter & 0.23 [0.18 - 0.30] \\ 
   & Summer & 0.32 [0.25 - 0.41] \\ 
  Creek chub & Winter & 0.25 [0.20 - 0.31] \\ 
   & Summer & 0.35 [0.28 - 0.44] \\ 
  Green sunfish & Winter & 0.13 [0.09 - 0.18] \\ 
   & Summer & 0.21 [0.15 - 0.27] \\ 
  Redbreast sunfish & Winter & 0.12 [0.07 - 0.21] \\ 
   & Summer & 0.29 [0.19 - 0.41] \\ 
   \hline
\end{tabular}
\end{table}


\newpage

\subsection{Movement model coefficients}

\small
{% latex table generated in R 4.4.2 by xtable 1.8-4 package
% Sun Jul 13 14:53:55 2025
\begin{table}[ht]
\centering
\caption{Parameter estimates of the movement model.
             Median estimates and their associated posterior probabilities are reported.} 
\label{tab:coefficients}
\begin{tabular}{llrrr}
  \hline
Species & Effect & Estimate & Pr(< 0) & Pr(> 0) \\ 
  \hline
Bluehead chub & Intercept & -0.25 & 0.90 & 0.10 \\ 
   & ln(Body length) & 0.02 & 0.47 & 0.53 \\ 
   & Habitat refuge area & -0.02 & 0.56 & 0.44 \\ 
   & Current velocity & -0.10 & 0.64 & 0.36 \\ 
   & Temperature & -0.38 & 0.98 & 0.02 \\ 
   & Density bluehead chub & 0.06 & 0.38 & 0.62 \\ 
   & Density bluehead chub & -0.02 & 0.55 & 0.45 \\ 
   & Density bluehead chub & -0.15 & 0.80 & 0.20 \\ 
   & Density redbreast sunfish & 0.04 & 0.38 & 0.62 \\ 
   & Recapture probability & 0.19 & 0.00 & 1.00 \\ 
   & d.f. in t distribution & 3.35 & 0.00 & 1.00 \\ 
  Creek chub & Intercept & -0.73 & 1.00 & 0.00 \\ 
   & ln(Body length) & 0.32 & 0.03 & 0.97 \\ 
   & Habitat refuge area & 0.06 & 0.28 & 0.72 \\ 
   & Current velocity & -0.09 & 0.85 & 0.15 \\ 
   & Temperature & -0.06 & 0.65 & 0.35 \\ 
   & Density bluehead chub & -0.15 & 0.92 & 0.08 \\ 
   & Density bluehead chub & 0.11 & 0.21 & 0.79 \\ 
   & Density bluehead chub & 0.19 & 0.12 & 0.88 \\ 
   & Density redbreast sunfish & -0.07 & 0.65 & 0.35 \\ 
   & Recapture probability & 0.13 & 0.00 & 1.00 \\ 
   & d.f. in t distribution & 3.13 & 0.00 & 1.00 \\ 
  Green sunfish & Intercept & -2.13 & 1.00 & 0.00 \\ 
   & ln(Body length) & 1.79 & 0.00 & 1.00 \\ 
   & Habitat refuge area & -0.55 & 0.97 & 0.02 \\ 
   & Current velocity & -0.08 & 0.69 & 0.31 \\ 
   & Temperature & -0.10 & 0.63 & 0.37 \\ 
   & Density bluehead chub & -0.95 & 1.00 & 0.00 \\ 
   & Density bluehead chub & 0.84 & 0.00 & 1.00 \\ 
   & Density bluehead chub & -0.25 & 0.78 & 0.22 \\ 
   & Density redbreast sunfish & 0.43 & 0.08 & 0.92 \\ 
   & Recapture probability & 0.13 & 0.00 & 1.00 \\ 
   & d.f. in t distribution & 3.07 & 0.00 & 1.00 \\ 
  Redbreast sunfish & Intercept & -0.34 & 0.93 & 0.07 \\ 
   & ln(Body length) & -0.27 & 0.89 & 0.11 \\ 
   & Habitat refuge area & -0.62 & 0.97 & 0.03 \\ 
   & Current velocity & -0.38 & 0.99 & 0.01 \\ 
   & Temperature & -0.14 & 0.76 & 0.24 \\ 
   & Density bluehead chub & -0.48 & 0.99 & 0.01 \\ 
   & Density bluehead chub & 0.44 & 0.01 & 0.99 \\ 
   & Density bluehead chub & -0.31 & 0.90 & 0.10 \\ 
   & Density redbreast sunfish & 0.18 & 0.20 & 0.80 \\ 
   & Recapture probability & 0.17 & 0.00 & 1.00 \\ 
   & d.f. in t distribution & 49.85 & 0.00 & 1.00 \\ 
   \hline
\end{tabular}
\end{table}
}


\end{document}